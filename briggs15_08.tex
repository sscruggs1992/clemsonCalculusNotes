\documentclass[mathNotesPreamble]{subfiles}
\begin{document}
\relscale{1.4} %TODO
\section{15.8: Lagrange Multipliers}

  \begin{defn*}[Parallel Gradients]
    Let $f$ be a differentiable function in a region of $\bbr^2$ that contains the smooth curve $C$ given by $g(x,y)=0$. Assume $f$ has a local extreme value on $C$ at a point $P(a,b)$. Then $\grad f(a,b)$ is orthogonal to the line tangent to $C$ at $P$. Assuming $\grad g(a,b)\neq \bfO$, it follows that there is a real number $\lambda$ (called a \textbf{Lagrange multiplier}) such that $\grad f(a,b)=\lambda \grad g(a,b)$.
  \end{defn*}

  \begin{thmBox*}[Procedure- Lagrange Multipliers: Absolute Extrema on Closed and Bounded Constraint Curves]
    Let the objective function $f$ and the constraint function $g$ be differentiable on a region $\bbr^2$ with $\grad g(x,y)\neq\bfO$ on the curve $g(x,y)=0$. To locate the absolute maximum and minimum values of $f$ subject to the constraint $g(x,y)=0$, carry out the following steps.
    \begin{enumerate}
      \item 
        Find the values of $x$, $y$, and $\lambda$ (if they exist) that satisfy the equations
          \[\grad f(x,y)=\lambda \grad g(x,y) \textnormal{ and } g(x,y)=0.\]
      \item 
        Evaluate $f$ at the values $(x,y)$ in Step 1 and at the endpoints of the constraint curve (if they exist). Select the largest and smallest corresponding function values. These values are the absolute maximum and minimum values of $f$ subject to the constraint.
    \end{enumerate}
  \end{thmBox*}

  \begin{thmBox*}[Procedure- Lagrange Multipliers: Absolute Extrema on Closed and Bounded Constraint Surfaces]
    Let $f$ and $g$ be differentiable on a region of $\bbr^3$ with $\grad g(x,y,z)\neq \bfO$ on the surface $g(x,y,z)=0$. To locate the absolute maximum and minimum values of $f$ subject to the constraint $g(x,y,z)=0$, carry out the following steps.
    \begin{enumerate}
      \item 
        Find the values of $x$, $y$, $z$, and $\lambda$ that satisfy the equations
          \[\grad f(x,y,z)=\lambda\grad g(x,y,z)\textnormal{ and } g(x,y,z)=0.\]
      \item 
        Among the points $(x,y,z)$ found in Step 1, select the largest and smallest corresponding function values. These values are the absolute maximum and minimum values of $f$ subject to the constraint.
    \end{enumerate}
  \end{thmBox*}



  \pagebreak
  
\end{document}