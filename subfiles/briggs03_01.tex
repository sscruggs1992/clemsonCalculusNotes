\documentclass[../mathNotesPreamble]{subfiles}
\begin{document}
%\relscale{1.4}
\section{3.1: Introducing the Derivative:}
  Recall that when given a distance function $s(t)$, the average velocity over the interval $\sbrkt{a,t}$ is
    $$v_\text{avg}=\dfrac{s(t)-s(a)}{t-a}$$
  and the instantaneous velocity is the limit of the average velocities as $t\to a$:
    $$v_\text{inst}=\lim_{t \to a}\dfrac{s(t)-s(a)}{t-a}$$
  Furthermore, the average velocity is the slope of the secant line through the points $\parens{a,s(a)}$ and $\parens{t,s(t)}$ and the instantaneous velocity is the slope of the tangent line at the point $\parens{a,s(a)}$.\hfill   
\textcolor{blue}{\underline{\url{https://www.desmos.com/calculator/08syaijrdo}}}

\vspace*{\stretch{1}}
\noindent
\begin{defn*}[Rate of Change and the Slope of the Tangent Line]\ 

  The \textbf{average rate of change} in $f$ on the interval $\sbrkt{a,x}$ is the slope of the corresponding secant line:
    $$m_\text{sec}=\frac{f(x)-f(a)}{x-a}$$
  The \textbf{instantaneous rate of change} in $f$ at $a$ is 
    $$m_\text{tan}=\lim_{x \to a} \frac{f(x)-f(a)}{x-a}$$
  which is also the \textbf{slope of the tangent line} at $(a,f(a))$, provided this limit exists. This \textbf{tangent line} is the unique line through $(a,f(a))$ with slope $m_\text{tan}$. Its equation is 
    $$y-f(a)=m_\text{tan}(x-a)$$
\end{defn*}
\pagebreak
\begin{ex*}
  Find an equation of the line tangent to the graph of $f(x)=\dfrac{3}{x}$ at $\parens{2,\dfrac{3}{2}}$.
  \vspace*{\stretch{1}}
\end{ex*}
\noindent
\begin{defn*}[Rate of Change and the Slope of the Tangent Line]\ 
  
  The \textbf{average rate of change} in $f$ on the interval $\sbrkt{a,a+h}$ is the slope of the corresponding secant line:
    $$m_\text{sec}=\frac{f(a+h)-f(a)}{h}.$$
  The \textbf{instantaneous rate of change} in $f$ at $a$ is
    $$m_\text{tan}=\lim_{h \to 0}\frac{f(a+h)-f(a)}{h},$$
  which is also the \textbf{slope of the tangent line} at $\parens{a,f(a)}$, provided this limit exists.
\end{defn*}

\pagebreak
\begin{ex*}
  Find an equation of the line tangent to the graph of $f(x)=x^3+4x$ at $(1,5)$.
  \vspace*{\stretch{1}}
\end{ex*}

\noindent
\begin{defn*}[The Derivative of a Function at a Point]\ 
  
  The \textbf{derivative of $f$ at $a$}, denoted $f'(a)$, is given by either of the two following limits, provided the limits exist and $a$ is in the domain of $f$:
    $$f'(a)=\lim_{x \to a}\frac{f(x)-f(a)}{x-a}\quad \text{(1) \ \ or } f'(a)=\lim_{h \to 0} \frac{f(a+h)-f(a)}{h}\quad \text{(2)}$$
  If $f'(a)$ exists, we say that $f$ is \textbf{differentiable at $a$}.
\end{defn*}

\pagebreak
\begin{ex*}
  Find an equation of the line tangent to the graph of $f(x)=\dfrac{8}{x^2}$ at $(2,2)$.
  \vspace*{\stretch{1}}
\end{ex*}
\begin{ex*}
  An equation of the line tangent to the graph of $f$ at the $(2,7)$ is $y=4x-1$. Find $f(2)$ and $f'(2)$.
  \vspace*{\stretch{1}}
\end{ex*}

\pagebreak
\begin{ex*}
  An equation of the line tangent to the graph of $g$ at $x=3$ is $y=5x+4$. Find $g(3)$ and $g'(3)$.
  \vspace*{\stretch{1}}
\end{ex*}
\begin{ex*}
  If $h(1)=2$ and $h'(1)=3$, find an equation of the line tangent to the graph of $h$ at $x=1$.
  \vspace*{\stretch{1}}
\end{ex*}
\begin{ex*}
  If $f'(-2)=7$, find an equation of the line tangent to the graph of $f$ at the point $(-2,4)$.
  \vspace*{\stretch{1}}
\end{ex*}
\pagebreak
\end{document}
