\documentclass[../mathNotesPreamble]{subfiles}
\begin{document}
%  \relscale{1.4}
  \section{10.1: An Overview of Sequences and Infinite Series}
    \begin{defn*}[Sequence]
      A \textbf{sequence} $\set{a_n}$ is an ordered list of numbers of the form
        \[\set{a_1,a_2,a_3,\dots,a_n,\dots}.\]
      A sequence may be generated by a \textbf{recurrence relation} of the form $a_\npo=f(a_n)$, for $n=1,2,3,\dots$, where $a_1$ is given. A sequence may also be defined with an \textbf{explicit formula} of the form $a_n=f(n)$, for $n=1,2,3,\dots$.
    \end{defn*}
    \begin{ex*}
      Consider the sequence $a_n=\frac{2^\npo}{2^n+1}$; Compute $a_1$, $a_2$, $a_3$, and $a_4$.
    \end{ex*}
    \vspace*{\stretch{1}}
    \pagebreak

    \begin{defn*}[Limit of a Sequence]
      If the terms of a sequence $\set{a_n}$ approach a unique number $L$ as $n$ increases--- that is, if $a_n$ can be made arbitrarily close to $L$ by taking $n$ sufficiently large--- then we say $\displaystyle\lim_{n\to \infty} a_n=L$ exists, and the sequence \textbf{converges} to $L$. If the terms of the sequence do not approach a single number as $n$ increases, the sequence has no limit, and the sequence \textbf{diverges}.
    \end{defn*}
    \begin{ex*}
      Determine if the sequence given by
        \[a_n=\frac{3+5n^2}{n+n^2}\]
      converges or diverges. If it converges, find the value that the sequence converges to.
    \end{ex*}
    \vspace*{\stretch{1}}

    \begin{ex*}
      Determine if the sequence given by
        \[a_n=(-1)^n\frac{3+5n^2}{n+n^2}\]
      converges or diverges. If it converges, find the value that the sequence converges to.
    \end{ex*}
    \vspace*{\stretch{1}}
    \pagebreak

    \begin{ex*}
      A ball is thrown upward to a height of 10 meters. After each bounce, the ball rebounds to $\sfrac{2}{3}$ of its previous height. Let $h_n$ be the height after the $n$th bounce. Find an explicit formula for the $n$th term of the sequence $\set{h_n}$.
    \end{ex*}
    \vspace*{\stretch{1}}
    \pagebreak

    \begin{defn*}[Infinite series]
      Given a sequence $\set{a_1, a_2, a_3,\dots}$, the sum of its terms
        \[a_1+a_2+a_3+\dots=\sum_{k=1}^\infty a_k\]
      is called an \textbf{infinite series}. The \textbf{sequence of partial sums} $\set{S_n}$ associated with this series has the terms
        \begin{align*}
          S_1&=a_1\\
          S_2&=a_1+a_2\\
          S_3&=a_1+a_2+a_3\\
          &\vdots\\
          S_n&=a_1+a_2+a_3+\dots+a_n=\sum_{k=1}^n a_k, &&\textnormal{ for } n=1,2,3,\dots.
        \end{align*}
      If the sequence of partial sums $\set{S_n}$ has a limit $L$, the infinite series \textbf{converges} to that limit, and we write
        \[\sum_{k=1}^\infty a_k=\lim_{n\to \infty}\underbrace{\sum_{k=1}^\infty a_k}_{S_n}=\lim_{n\to \infty}S_n=L.\]
      If the sequence of partial sums diverges, the infinite series also \textbf{diverges}.
    \end{defn*}
    \vspace*{\stretch{1}}
    \pagebreak

    \begin{ex*}
      Consider the infinite series $4+0.9+0.09+0.009+\dots$. Compute $S_1$, $S_2$, $S_3$, and $S_4$. What is the value of this series?
    \end{ex*}
    \vspace*{\stretch{1}}
    \pagebreak

    \begin{ex*}
      A sequence $\set{a_n}$ has partial sums given by the formula $S_n=5-\frac{1}{\sqrt{n}}$. 
    \end{ex*}
    \begin{tasks}[after-item-skip=\stretch{1}, label=, item-indent=0pt](1)
      \task What is the value of the series $\displaystyle\sum_{n=1}^\infty a_n$?
      \task What is the formula for $a_n$?
      \task What is the limit $\displaystyle\lim_{n\to \infty} a_n$?
    \end{tasks}
    \vspace*{\stretch{1}}
    \pagebreak

\end{document}
