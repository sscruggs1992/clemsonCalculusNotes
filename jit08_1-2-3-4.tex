\documentclass[mathNotesPreamble]{subfiles}
\begin{document}
%\relscale{1.4}
\section{JIT 8.1: Definitions of Logarithms}
  \begin{defn*}[Logarithmic Function Base $b$]
    For any base $b>0$, with $b\neq 1$, the \textbf{logarithmic function base $b$}, denoted $y~=~\log_b(x)$, is the inverse of the exponential function $y=b^x$. The inverse of the natural exponential function with base $b=e$ is the \textbf{natural logarithm function}, denoted $y=\ln(x)$.
  \end{defn*}
  
  \noindent
  \begin{minipage}{0.5\linewidth}
    \begin{center}
      \begin{tikzpicture}
        \begin{axis}[
          axis lines=center,
          axis line style={->},
          axis equal,
          xmin=-4.25, xmax=4.25,
          ymin=-4.25, ymax=4.25,
          xmajorticks=false,
          ymajorticks=false,
          ticklabel style={font=\tiny,inner sep=0.75pt,fill=white},
          xlabel=$x$, xlabel style={at={(ticklabel* cs:1)},anchor=north west},
          ylabel=$y$, ylabel style={at={(ticklabel* cs:1)},anchor=south west},
          every axis plot/.append style={line width=1.25pt}
          ]
          \addplot[<->] expression[domain=-5.125:1.45, ClemsonPurple, samples=100] {e^x} node[black, right, pos=0.95, fill=white, xshift=3pt] {$a^x$};
          \addplot[<->] expression[domain=0.0142:5.1, ClemsonOrange,  samples=100] {ln(x)} node[black, above, pos=0.875, fill=white, yshift=5pt] {$\log_a(x)$};
          \addplot[dashed] expression[domain=-5.5:5.5, black!50] {x};
        \end{axis}
      \end{tikzpicture}
    \end{center}
  \end{minipage}%
  \begin{minipage}{0.5\linewidth}
    \textit{Note:}
    \begin{center}
      \begin{tabular}{@{}lCC@{}}\toprule
        & a^x& \log_a(x)\\\midrule
        Domain:& (-\infty,\infty)& (0,\infty)\\
        Range: & (0,\infty)& (-\infty,\infty)\\\bottomrule
      \end{tabular}
    \end{center}
    \vspace*{55pt}
  \end{minipage}%
  
  \begin{center}
    \fbox{\parbox{0.37\linewidth}{
    \centering $y=b^x\quad\iff\quad\log_b(y)=x$
    
    Think ``the base stays the base''
    }}
  \end{center}
  \begin{ex*}
    Evaluate:
    \begin{extasks}(2)
      \task $\log_9(81)$
      \task $\log_3(\sqrt3)$\\
      \task $\log_{\frac{1}{2}}(8)$
      \task $\parens{\log_5(5\inv[3])}^2$\\
    \end{extasks}
  \end{ex*}
  \pagebreak
  
  \textit{Note:} In this course, the \textbf{common logarithm} is $\log_{10}(x)$ and is denoted by $\log(x)$. 
  
  -- Sometimes other disciplines use $\log(x)$ to represent other bases.
  \begin{ex*}
    Evaluate:
    \begin{extasks}(2)
      \task $\log 100000$
      \task $\log \frac{1}{1000}$\\
    \end{extasks}
  \end{ex*}
\section{JIT 8.2: Logs as Inverses of Exponential Functions}
  Recall that for a function $f$ and its inverse $g$:
  \begin{tasks}[style=itemize](2)
    \task $f\parens{g(x)}=x$
    \task Domain of $f$=Range of $g$
    \task $g\parens{f(x)}=x$
    \task Domain of $g$=Range of $f$\\
  \end{tasks}
  \begin{center}
    \fbox{\parbox{0.9\linewidth}{\textbf{Inverse Relations for Exponential and Logarithmic Functions}
    
    For any base $b>0$, with $b\neq 1$, the following inverse relations hold:
      $$b^{\log_bx}=x \hspace*{0.2\linewidth}\log_b(b^x)=x,\text{ for all real values of }x$$
    }}
  \end{center}
  \begin{ex*}
    Evaluate:
    \begin{extasks}(3)
      \task $2^{\log_28}$
      \task $\log_bb^\pi$
      \task $\log10^3$
    \end{extasks}
  \end{ex*}
  \vfill
  \pagebreak
  
\section{JIT 8.3: Laws of Logarithms}
  \begin{ex*}
    Write each expression in terms of one logarithm:\\
    
    \noindent
    \begin{minipage}{0.6\linewidth}
      \begin{itemize}
        \item[] $\log_2 6-\log_2 15+\log_2 20$\\[50pt]
        \item[] $\log_3 100-\parens{\log_3 18+\log_3 50}$\\[50pt]
      \end{itemize}
    \end{minipage}%
    \begin{minipage}{0.4\linewidth}
      \begin{flushright}
        \fbox{\parbox{0.9\linewidth}{\textbf{Laws of Logarithms}
      
          For $x,y>0$:
            \begin{enumerate}
              \item $\log_a(xy)=\log_a(x)+\log_a(y)$
              \item $\log_a\parens{\frac{x}{y}}=\log_a(x)-\log_a(y)$
              \item $\log_a(x^r)=r\log_a(x)$
              \item $\log_a(1)=0$
              \item $\log_a(x)=\dfrac{\log_b(x)}{\log_b(a)}$
            \end{enumerate}
          }}
      \end{flushright}
    \end{minipage}
  \end{ex*}
  \pagebreak
  \begin{ex*}
    Solve each equation checking for extraneous solutions:
    \begin{enumerate}[label=, itemsep=\stretch{1}]
      \item $\log_{64}x^2=\frac{1}{3}$
      \item $\log(3x+2)+\log(x-1)=2$
      \item $\log_2 x^2-\log_2(3x-8)=2$
      \item $\log_4 x-\log_4(x-1)=\frac{1}{2}$
      \item $\log_3(x+6)-\log_3(x-6)=2$
      \item $\log_3(x^2-5)=2$
    \end{enumerate}
  \end{ex*}
  \vspace*{\stretch{1}}
  \pagebreak
\section{JIT 8.4: The Natural Logarithm}
  \noindent
  \begin{minipage}{0.725\linewidth}
    \begin{defn*}
      The \textbf{Natural Logarithmic Function} uses base $e$ and is denoted $\ln(x)=\log_e x$.
    \end{defn*}
    \begin{center}
      \textit{Note:} The natural log is the inverse of $e^x$:
      $$\ln(x)=y\ \iff\ e^y=x$$
    \end{center}
    \vspace*{5pt}
  \end{minipage}%
  \begin{minipage}{0.275\linewidth}
    \begin{flushright}
      \begin{tikzpicture}[scale=0.65]
        \begin{axis}[
          axis lines=center,
          axis line style={->},
          axis equal,
          xmin=-4.25, xmax=4.25,
          ymin=-4.25, ymax=4.25,
          xmajorticks=false,
          ymajorticks=false,
          ticklabel style={font=\tiny,inner sep=0.75pt,fill=white},
          xlabel=$x$, xlabel style={at={(ticklabel* cs:1)},anchor=north west},
          ylabel=$y$, ylabel style={at={(ticklabel* cs:1)},anchor=south west},
          every axis plot/.append style={line width=1.25pt}
          ]
          \addplot[<->] expression[domain=-5.125:1.45, ClemsonPurple, samples=100] {e^x} node[black, right, pos=0.95, fill=white, xshift=3pt] {$e^x$};
          \addplot[<->] expression[domain=0.0142:5.1, ClemsonOrange,  samples=100] {ln(x)} node[black, above, pos=0.875, fill=white, yshift=5pt] {$\ln(x)$};
          \addplot[dashed] expression[domain=-5.5:5.5, black!50] {x};
        \end{axis}
      \end{tikzpicture}
    \end{flushright}
  \end{minipage}
  \begin{center}
    \fbox{\parbox{0.9\linewidth}{\textbf{Inverse Properties for $a^x$ and $\log_a x$}
      \begin{center}
        \begin{tabular}{l@{\quad}*{2}{R@{\ =\ }L@{\qquad}}L}
          Base $a$:& a^{\log_a x}&x,& \log_a a^x&x,& a>0, a\neq 1, x>0\\
          Base $e$:& e^{\ln x}&x,   & \ln e^x   &x,& x>0
        \end{tabular}
      \end{center}
    }}
  \end{center}
  
  \begin{ex*}
    Graph the following functions:
    
  \end{ex*}
  \uplevel{\centering
    \begin{tikzpicture}[scale=0.74]
      \begin{groupplot}[
        group style={group size=3 by 2, horizontal sep=2cm, vertical sep=5cm},
        axis lines=center,
        axis line style={->},
        axis equal,
        xmin=-4.25, xmax=4.25,
        ymin=-4.25, ymax=4.25,
        xmajorticks=false,
        ymajorticks=false,
        ticklabel style={font=\tiny,inner sep=0.75pt,fill=white},
        xlabel=$x$, xlabel style={at={(ticklabel* cs:1)},anchor=north west},
        ylabel style={at={(ticklabel* cs:1.05), font=\Huge},anchor=south},
        every axis plot/.append style={line width=1.25pt}
        ]
        \nextgroupplot[ylabel=$e^{x+2}$]
        \nextgroupplot[ylabel=$e^x+2$]
        \nextgroupplot[ylabel=$e^{x-1}+2$] 
        \nextgroupplot[ylabel=$\ln(x+2)$]
        \nextgroupplot[ylabel=$\ln x+2$]
        \nextgroupplot[ylabel=$\ln(x+1)-2$]
      \end{groupplot}
    \end{tikzpicture}}
  \pagebreak
  \begin{ex*}
    Simplify
    \begin{extasks}(2)
      \task $e\inv[\ln 0.3]$
      \task $e^{\ln \pi x-\ln 2}$\\[20pt]
      \task $\ln\parens{\frac{1}{e}}$
      \task $e^{4\ln x}$\\[20pt]  
    \end{extasks}
  \end{ex*}
  \begin{ex*}
    Write each expression in terms of one logarithm:\\
    
    \noindent
    \begin{minipage}{0.55\linewidth}
      \begin{extasks}(1)
        \task $\ln(a+b)+\ln(a-b)-2\ln c$\\[70pt]
        \task $\frac{1}{3}\ln(x+2)^3 +\frac{1}{2}\sbrkt{\ln x-\ln(x^2+3x+2)^2}$\\[30pt]
      \end{extasks}
    \end{minipage}%
    \begin{minipage}{0.45\linewidth}
      \begin{flushright}
        \fbox{\parbox{0.9\linewidth}{\textbf{Laws of the Natural Logarithm}
      
          For $x,y>0$:
            \begin{enumerate}
              \item $\ln(xy)=\ln(x)+\ln(y)$
              \item $\ln\parens{\frac{x}{y}}=\ln(x)-\ln(y)$
              \item $\ln(x^r)=r\ln(x)$
              \item $\ln(1)=0$
              \item $\log_a(x)=\dfrac{\ln(x)}{\ln(a)}$
            \end{enumerate}
          }}
      \end{flushright}
    \end{minipage}
    \vspace*{\stretch{1}}
  \end{ex*}
  \begin{center}
    \fbox{\parbox{0.9\linewidth}{
      \textit{Note:} Many common mistakes come from using the logarithm rules incorrectly:
      $$\ln A-\ln B\neq \dfrac{\ln A}{\ln B}\qquad \ln(A+B)\neq\ln(A)\ln(B)$$
    }}
  \end{center}
  \pagebreak
  \begin{ex*}
    Solve:
    \begin{extasks}(2)
      \task $2^x=55$
      \task $5^{3x}=29$\\[40pt]
      \task $e^{2x}-5e^x-14=0$
      \task $4e^{2x}-7e^x=15$\\[40pt]
    \end{extasks}
  \end{ex*}
  \begin{ex*}
    Solve for $y$ in terms of $x$:
    \begin{extasks}(2)
      \task $\ln(y-40)=5x$
      \task $\ln(y^2-1)-\ln(y+1)=\ln(\sin x)$
    \end{extasks}
  \end{ex*}
  \vfill
  \begin{ex*}
    Solve:
    \begin{tasks}[after-item-skip=\stretch{1}](2)
      \task[] $\ln(t)+\ln(t^2)=6$
      \task[] $e^{x^2+2x-3}=1$
      \task[] $\ln x=-1$
      \task[] $e\inv[0.3t]=27$
    \end{tasks}
  \end{ex*}
  \vspace*{\stretch{1}}
  \pagebreak
  
\end{document}
