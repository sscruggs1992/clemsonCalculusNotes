\documentclass[mathNotesPreamble]{subfiles}
\begin{document}
\relscale{1.4} %TODO
\section{15.3: Partial Derivatives}

  \begin{defn*}[Partial Derivatives]
    The \textbf{partial derivative of $f$ with respect to $x$ at the point $(a,b)$ is}
      \[f_x(a,b)=\lim_{h\to 0} \frac{f(a+h,b)-f(a,b)}{h}.\]
    The \textbf{partial derivative of $f$ with respect to $y$ at the point $(a,b)$ is}
      \[f_y(a,b)=\lim_{h\to 0} \frac{f(a,b+h)-f(a,b)}{h},\]
    provided these limits exist.
  \end{defn*}

  \noindent
  \fbox{\parbox{0.9875\linewidth}{
    \textbf{Theorem 15.4: (Clairut) Equality of Mixed Partial Derivatives}
    Assume $f$ is defined on an open set $D$ of $\bbr^2$, and that $f_{xy}$ and $f_{yx}$ are continuous throughout $D$. Then $f_{xy}=f_{yx}$ at all points of $D$.
  }}

  \begin{defn*}[Differentiability]
    The function $z=f(x,y)$ is \textbf{differentiable at $(a,b)$} provided $f_x(a,b)$ and $f_y(a,b)$ exist and the change $\Delta z=f(a+\Delta x, b+\Delta y)-f(a,b)$ equals
      \[\Delta z=f_x(a,b)\Delta x+ f_y(a,b)\Delta y=\eps_1\Delta x+\eps_2\Delta y,\]
    where for fixed $a$ and $b$, $\eps_1$ and $\eps_2$ are functions that depend only on $\Delta x$ and $\delta y$, with $\parens{\eps_1,\eps_2}\to (0,0)$ as $\parens{\Delta x,\Delta y}\to(0,0)$. A function is \textbf{differentiable} on an open set $R$ if it is differentiable at every point of $R$.
  \end{defn*}

  \noindent
  \fbox{\parbox{0.9875\linewidth}{
    \textbf{Theorem 15.5: Conditions for Differentiability}
    
    Suppose the function $f$ has partial derivatives $f_x$ and $f_y$ defined on an open set containing $(a,b)$, with $f_x$ and $f_y$ continuous at $(a,b)$. Then $f$ is differentiable at $(a,b)$.
  }}

  \noindent
  \fbox{\parbox{0.9875\linewidth}{
    \textbf{Theorem 15.6: Differentiable Implies Continuous}
    
    If a function $f$ is differentiable at $(a,b)$, then it is continuous at $(a,b)$.
  }}

  \pagebreak
  
\end{document}