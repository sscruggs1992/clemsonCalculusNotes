\documentclass[mathNotesPreamble]{subfiles}
\begin{document}
\relscale{1.4} %TODO
\section{14.5: Curvature and Normal Vectors:}

  There are two ways to acceleration:
  \begin{itemize}
    \item 
      change in speed
    \item 
      change in direction
  \end{itemize}
  The change in direction is referred to as \textit{curvature}. Recall that if we have a smooth curve $\vecr(t)$, the unit tangent vector is
    \[\mathbf T(t)=\frac{\vecr'(t)}{\abs{\vecr'(t)}}=\frac{\vecv(t)}{\abs{\vecv(t)}}\]

  Specifically, \textit{curvature} of the curve is the magnitude of the rate at which $\mathbf T$ changes with respect to arc length.
  \vspace*{\stretch{1}}

  \begin{center}
    \includegraphics[width=0.8\linewidth]{images/briggs_14_05/fig14_29}
  \end{center}
  \vspace*{\stretch{1}}

  \begin{defn*}[Curvature]
    Let $\vecr$ describe a smooth parameterized curve. If $s$ denotes arc length and $\mathbf T=\vecr'/\abs{\vecr'}$ is the unit tangent vector, the \textbf{curvature} is $\ds\kappa(s)=\abs{\frac{d\mathbf T}{ds}}$.
  \end{defn*}
  \pagebreak

  

  \noindent
  \fbox{\parbox{0.9875\linewidth}{
    \textbf{Theorem 14.4: Curvature Formula}\\
    Let $\vecr(t)$ describe a smooth parameterized curve, where $t$ is any parameter. If $\vecv=\vecr'$ is the velocity and $\mathbf T$ is the unit tangent vector, then the curvature is
      \[\kappa(t)=\frac{1}{\abs{\vecv}}\abs{\frac{d\mathbf T}{dt}}=\frac{\abs{\mathbf T'(t)}}{\abs{\vecr'(t)}}.\]
  }}

  \noindent
  \fbox{\parbox{0.9875\linewidth}{
    \textbf{Theorem 14.5: Alternative Curvature Formula}\\
    Let $\vecr$ be the position of an object moving on a smooth curve. The \textbf{curvature} at a point on the curve is
      \[\kappa=\frac{\abs{\vecv\times\mathbf a}}{\abs{\vecv}^3},\]
    where $\vecv=\vecr'$ is the velocity and $\mathbf a=\vecv'$ is the acceleration.
  }}

  \begin{defn*}[Principal Unit Normal Vector]
    Let $\vecr$ describe a smooth curve parameterized by arc length. The \textbf{principal unit normal vector} at a point $P$ on the curve at which $\kappa\neq 0$ is
      \[\mathbf N(s)=\frac{d\mathbf T/ds}{\abs{d\mathbf T/ds}}=\frac{1}{\kappa}\frac{d\mathbf T}{ds}.\]
    For other parameters, we use the equivalent formula
      \[\mathbf N(t)=\frac{d\mathbf T/dt}{\abs{d\mathbf T/dt}},\]
    evaluated at the value of $t$ corresponding to $P$.
  \end{defn*}

  \noindent
  \fbox{\parbox{0.9875\linewidth}{
    \textbf{Theorem 14.6: Properties of the Principal Unit Normal Vector}\\
    Let $\vecr$ describe a smooth parameterized curve with unit tangent vector $\mathbf T$ and principal unit normal vector $\mathbf N$.
    \begin{enumerate}
      \item 
        $\mathbf T$ and $\mathbf N$ are orthogonal at all points of the curve; that is, $\mathbf T\cdot\mathbf N=0$ at all points where $\mathbf N$ is defined.
      \item 
        The principal unit normal vector points to the inside of the curve -- in the direction that the curve is turning.
    \end{enumerate}
  }}
  
  \noindent
  \fbox{\parbox{0.9875\linewidth}{
    \textbf{Theorem 14.7: Tangential and Normal Components of the Acceleration}\\
    The acceleration vector of an object moving in space along a smooth curve has the following representation in terms of its \textbf{tangential component} $a_T$ (in the direction of $\mathbf T$) and its \textbf{normal component} $a_N$ (in the direction of $\mathbf N$):
      \[\mathbf a=a_N\mathbf N+a_T\mathbf T,\]
    where $a_N=\kappa\abs{\vecv}^2=\ds\frac{\abs{\vecv\times\mathbf a}}{\abs{\vecv}}$ and $a_T=\ds\frac{d^2s}{dt^2}$.
  }}

  \begin{defn*}[Unit Binormal Vector and Torsion]
    Let $C$ be a smooth parameterized curve with unit tangent and principal unit normal vectors $\mathbf T$ and $\mathbf N$, respectively. Then at each point of the curve at which the curvature is nonzero, the \textbf{unit binomial vector} is
      \[\mathbf B=\mathbf T\times\mathbf N,\]
    and the \textbf{torsion} is
      \[\tau=-\frac{d\mathbf B}{ds}\cdot\mathbf N\]
  \end{defn*}

  \fbox{\parbox{0.9875\linewidth}{
    \textbf{Summary: Formula for Curves in Space}\\
    
  }}

  \pagebreak
  
\end{document}